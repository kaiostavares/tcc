% ----------------------------------------------------------
\xchapter{Introdução}{}
\label{cap:introducao}
% ----------------------------------------------------------

A verificação da conformidade entre os requisitos de software e sua implementação constitui um desafio persistente na Engenharia de Software (ES), sobretudo em organizações que buscam entregar sistemas com agilidade, sem comprometer a qualidade. A crescente pressão por entregas rápidas, aliada à complexidade dos sistemas modernos, tem imposto uma sobrecarga significativa às equipes de qualidade, responsáveis por assegurar que as funcionalidades entregues estejam em consonância com as especificações previamente definidas. Esse cenário torna-se ainda mais crítico em projetos com alta rotatividade de desenvolvedores ou com documentação deficiente, nos quais a rastreabilidade entre o que foi solicitado e o que foi efetivamente entregue tende a se perder ao longo do ciclo de desenvolvimento.

Nesse contexto, o retrabalho torna-se uma consequência frequente, uma vez que a detecção precoce de falhas nem sempre é plenamente alcançada, o que compromete a confiabilidade e a sustentabilidade do produto. Para mitigar tais desafios, é comum que desenvolvedores e equipes de software utilizem ferramentas que contribuam para otimizar e agilizar suas atividades. Estratégias como testes automatizados, \textit{linters} e analisadores estáticos de código desempenham papel relevante nesse cenário. No entanto, tais abordagens tendem a se concentrar predominantemente em aspectos técnicos e sintáticos do código-fonte, sendo, por vezes, insuficientes para avaliar a aderência funcional da implementação às regras de negócio previamente estabelecidas, sobretudo no que diz respeito às boas práticas e ao contexto dos requisitos.

Diante disso, propõe-se o uso de Large Language Models (LLMs) como ferramenta auxiliar na análise de conformidade entre requisitos e implementação, ainda durante a etapa de desenvolvimento. A ideia central consiste em empregar modelos de linguagem treinados para interpretar descrições em linguagem natural, cruzando-as com trechos de código-fonte extraídos dos repositórios do projeto, de modo a identificar possíveis desvios, omissões ou implementações inconsistentes. Essa abordagem visa não apenas aumentar a rastreabilidade semântica entre os artefatos, mas também reduzir a dependência exclusiva da equipe de qualidade (Quality Assurance — QA) na validação contínua dos requisitos, aliviando sua carga operacional.

Do ponto de vista metodológico, a proposta envolve um fluxo estruturado que parte da extração de conteúdo de repositórios Git, onde se encontram os arquivos de código-fonte, seguido da aplicação de técnicas de engenharia de prompt para a construção de instruções capazes de orientar a LLM na análise. Tais prompts são refinados iterativamente com base nos resultados obtidos, incorporando estratégias como \textit{few-shot prompting} e \textit{chain-of-thought}, com o objetivo de aumentar a robustez das respostas. Diferentes modelos são testados com o mesmo prompt, permitindo observar variações no desempenho e compreender como seus parâmetros e arquiteturas influenciam a capacidade de interpretar requisitos e gerar inferências corretas.

Assim, esta pesquisa tem como objetivo investigar em que medida a aplicação de modelos de linguagem de grande escala pode contribuir, de forma prática e objetiva, para a redução do retrabalho, o fortalecimento da rastreabilidade e o aprimoramento da qualidade em ambientes de desenvolvimento de software. Mais do que propor uma solução teórica, este estudo ancora-se em um cenário simulado com base em um projeto real - o \textit{Nivelamento Online}, desenvolvido pela empresa Exatamente Soluções Educacionais, no qual são exploradas implementações concretas e casos de uso reais. Tal abordagem permite observar a eficácia da solução proposta sob diferentes perspectivas. No aspecto técnico, considera-se a atuação da LLM na interpretação e verificação do código. No aspecto processual, avalia-se a integração dessa verificação aos fluxos de desenvolvimento. Já no aspecto metodológico, são examinadas as etapas de engenharia de prompt e a análise comparativa entre diferentes modelos extraindo diferentes métricas.

\section{Objetivos}

\subsection{Objetivo Geral}
O presente trabalho tem como objetivo geral investigar sistematicamente a aplicação de LLMs na verificação automatizada da conformidade entre requisitos de software, descritos em linguagem natural, e sua respectiva implementação no código-fonte. Busca-se, com isso, desenvolver e validar um método capaz de reduzir o retrabalho, fortalecer a rastreabilidade e apoiar a garantia da qualidade em ambientes reais de desenvolvimento.

\subsection{Objetivos Específicos}
\begin{enumerate}
    \item Avaliar o desempenho de diferentes LLMs na tarefa de interpretar requisitos e localizar os trechos de código que atendam às funcionalidades descritas.
    
    \item Analisar o impacto de técnicas de engenharia de prompt na precisão e na consistência das respostas geradas pelos modelos.
    
    \item Comparar a performance de múltiplos modelos frente a um mesmo conjunto de requisitos e código-fonte, observando suas variações.
    
    \item Identificar as limitações práticas e os desafios operacionais no uso de LLMs em fluxos de desenvolvimento de software, considerando aspectos como a ambiguidade dos requisitos e a complexidade dos repositórios.
    
    \item Sistematizar um conjunto de boas práticas e recomendações para o uso eficaz de LLMs em atividades de verificação e validação de requisitos no ciclo de vida do software.
\end{enumerate}

\section{Justificativa}

A aplicação de LLMs na engenharia de software emerge como uma abordagem promissora para solucionar desafios recorrentes na verificação da conformidade entre requisitos e implementação. Tradicionalmente, este processo depende de um esforço manual significativo por parte das equipes de qualidade, o que pode comprometer a agilidade do desenvolvimento e elevar o risco de falhas não detectadas, especialmente em projetos com alta rotatividade de pessoal, documentação limitada ou prazos restritos.

Embora ferramentas como testes automatizados, \textit{linters} e analisadores estáticos de código exerçam um papel fundamental na garantia da qualidade, sua atuação concentra-se, em grande medida, em aspectos sintáticos e estruturais. Nesse contexto, a ausência de soluções capazes de interpretar a lógica de negócio expressa nos requisitos e relacioná-la semanticamente à implementação constitui uma lacuna expressiva no estado da prática, limitando a automação de validações mais complexas.

Portanto, este trabalho se justifica pela oportunidade de inovação ao investigar o potencial dos LLMs para preencher essa lacuna, atuando diretamente sobre os repositórios de código durante a fase de desenvolvimento. Ao explorar um fluxo que envolve a extração de artefatos, a construção e o refinamento de \textit{prompts}, e a análise comparativa entre diferentes modelos, a pesquisa busca demonstrar como tais tecnologias podem contribuir, de forma prática e objetiva, para a melhoria da rastreabilidade, a antecipação de falhas e a redução do retrabalho.

Ademais, a aplicação da metodologia em um projeto real, desenvolvido na empresa Exatamente Soluções Educacionais, confere relevância prática à investigação. Tal abordagem permite observar os efeitos da proposta em um cenário com demandas concretas, oferecendo subsídios valiosos para a adoção responsável e eficaz dessas ferramentas tanto em ambientes produtivos quanto no meio acadêmico.

\section{Estrutura do Trabalho}
Este trabalho está estruturado em cinco capítulos, além desta Introdução. Os Capítulos 2 e 3 são dedicados à revisão bibliográfica. O Capítulo 2 discute fundamentos de Engenharia de Software relevantes à pesquisa, como engenharia de requisitos, rastreabilidade, garantia da qualidade e práticas de verificação e validação. O Capítulo 3, por sua vez, revisa conceitos associados ao Aprendizado de Máquina, abordando modelos de linguagem de larga escala (LLMs), arquiteturas baseadas em transformadores, técnicas de \textit{prompting} (como \textit{few-shot} e \textit{chain-of-thought}), métricas, desafios e estudos correlatos que aplicam LLMs no contexto da Engenharia de Software.

O Capítulo 4 descreve a metodologia adotada, detalhando o fluxo proposto para a aplicação dos LLMs na verificação da conformidade entre requisitos e código-fonte. O Capítulo 5 apresenta os resultados obtidos a partir da aplicação do método em um cenário real, discutindo as métricas avaliadas, os achados mais relevantes e as limitações observadas. Por fim, o Capítulo 6 conclui o trabalho, sintetizando as contribuições alcançadas e apontando direções para investigações futuras.