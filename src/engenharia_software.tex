\xchapter{Engenharia de Software}{}

A Engenharia de Software é uma disciplina fundamental para a construção sistemática, disciplinada e mensurável de sistemas computacionais. De acordo com \citeonline{sommerville2011}, a ES não se restringe à atividade de programação, mas abrange todas as etapas do ciclo de vida do software, desde a especificação inicial até a manutenção de sistemas em operação. Seu objetivo central é assegurar que os sistemas sejam desenvolvidos de maneira confiável, econômica e em conformidade com as necessidades dos usuários.

Ademais, \citeonline{pressman2016} complementa essa visão ao descrever a ES como uma tecnologia alicerçada em camadas. Em sua base está o compromisso contínuo com a qualidade, sobre o qual se apoiam o processo, os métodos e as ferramentas. O processo define as atividades e práticas que orientam o desenvolvimento; os métodos oferecem o suporte técnico para análise, projeto, implementação e testes; e as ferramentas fornecem automação para aumentar a produtividade e a consistência.

Nesse sentido, ambos os autores convergem ao destacar que a Engenharia de Software é uma prática sócio-técnica, que integra dimensões técnicas, humanas e organizacionais. Isso implica reconhecer que fatores como comunicação eficaz, gerenciamento de riscos, colaboração entre equipes e adaptabilidade são tão cruciais quanto as próprias técnicas de desenvolvimento.

Sob essa ótica, este trabalho se fundamenta nos princípios da Engenharia de Software para investigar a aplicação de \textit{Large Language Models}  no apoio à análise de conformidade entre requisitos e código. A pesquisa propõe-se a conectar os desafios clássicos da área, especialmente a rastreabilidade semântica entre especificações em linguagem natural e suas implementações com as oportunidades oferecidas por tecnologias emergentes, buscando soluções mais eficientes e automatizadas.

\section{Engenharia de Requisitos}

A Engenharia de Requisitos é reconhecida como uma das atividades centrais da Engenharia de Software, sendo responsável por estabelecer a base sobre a qual o sistema será desenvolvido. Sua finalidade é transformar as necessidades dos \textit{stakeholders} em especificações compreensíveis, consistentes e verificáveis, criando um elo entre os objetivos de negócio e a solução técnica. Como destaca \citeonline{sommerville2011}, a Engenharia de Requisitos é um processo iterativo que envolve identificar os serviços que o sistema deve oferecer e as restrições que condicionam seu funcionamento.

A literatura aponta que falhas nessa etapa representam uma das principais causas de insucesso em projetos de software. \citeonline{pressman2016} enfatiza que compreender o que deve ser construído é, frequentemente, mais desafiador do que a própria construção, pois clientes e usuários raramente possuem uma visão completa de suas necessidades ou as expressam de forma inequívoca. Ademais, os requisitos são naturalmente voláteis, sofrendo alterações ao longo do projeto. Desse modo, a Engenharia de Requisitos deve ser entendida como um processo contínuo de comunicação, análise e validação.

Do ponto de vista processual, a Engenharia de Requisitos compreende um conjunto de atividades integradas. Inicia-se pela concepção, que define o escopo preliminar do problema. Segue-se para a elicitação, que consiste na coleta estruturada de informações por meio de técnicas como entrevistas, prototipação e análise documental. A fase de análise e elaboração busca refinar os dados obtidos, identificando inconsistências e ambiguidades. Dada a diversidade de interesses, a negociação torna-se indispensável para conciliar demandas conflitantes dentro das restrições do projeto.

Uma vez definidas as prioridades, procede-se à especificação, formalizada em documentos como a \textit{Software Requirements Specification} (SRS), que registram requisitos funcionais e não funcionais. A etapa seguinte, a validação, verifica se as especificações refletem as necessidades dos \textit{stakeholders} e se são completas e testáveis, utilizando técnicas como revisões e simulações. Por fim, a gestão de requisitos garante o acompanhamento das mudanças ao longo do ciclo de vida, mantendo vínculos claros entre requisitos, implementações e testes para preservar a rastreabilidade.

O Guia PMBOK \textit{\cite{pmi2008}}, por sua vez, insere a Engenharia de Requisitos no contexto do gerenciamento do escopo do projeto, destacando que cada necessidade deve ser documentada e vinculada a entregáveis mensuráveis. A rastreabilidade, nesse sentido, é um mecanismo essencial que assegura a correta implementação e verificação dos requisitos, promovendo o alinhamento entre planejamento, execução e controle.

Assim, a Engenharia de Requisitos não se limita a uma etapa preliminar de coleta de informações, mas constitui um processo contínuo e estruturado, que articula comunicação entre \textit{stakeholders}, documentação formal, validação sistemática e gestão de mudanças. Ao desempenhar esse papel central, garante que os sistemas de software sejam construídos de forma a atender às expectativas dos usuários, respeitar restrições operacionais e oferecer qualidade consistente ao longo de seu ciclo de vida.

\section{Gerenciamento de Prazos e Custos}

O gerenciamento de prazos e custos constitui uma das áreas mais sensíveis da Engenharia de Software, dada a natureza abstrata do produto e a volatilidade dos requisitos. A ausência de elementos físicos tangíveis, somada às dificuldades de mensuração do esforço intelectual, torna a previsão de tempo e orçamento uma atividade inerentemente incerta e sujeita a desvios.

Segundo \citeonline{sommerville2011}, abordagens de planejamento rígido apresentam limitações em projetos de software, especialmente em contextos onde mudanças são inevitáveis. Nesses casos, modelos iterativos e incrementais, que permitem reavaliações contínuas, proporcionam maior capacidade de resposta a imprevistos. O autor destaca, ainda, a importância do replanejamento constante e do envolvimento dos \textit{stakeholders} para garantir o alinhamento entre expectativas e entregáveis.

Do ponto de vista operacional, o planejamento envolve a definição de atividades, a estimativa de esforço, a alocação de recursos e o estabelecimento de marcos de controle. \citeonline{pressman2016} enfatiza que a precisão das estimativas depende da maturidade organizacional e da disponibilidade de dados históricos. Em ambientes estruturados, técnicas como a Análise de Pontos de Função e o modelo COCOMO (\textit{Constructive Cost Model}) são utilizadas para fornecer estimativas quantitativas, permitindo simular cenários de custo e cronograma com base em parâmetros como complexidade e capacidade da equipe.

A comunicação eficaz também é um pilar deste processo. Conforme argumenta \citeonline{sommerville2011}, falhas na comunicação figuram entre as principais causas de atrasos e estouros orçamentários. A revisão periódica dos planos, o registro transparente de mudanças e a gestão ativa das expectativas são, portanto, cruciais para a manutenção do controle do projeto.

Nesse contexto, a conexão entre a conformidade de requisitos e o gerenciamento de prazos e custos torna-se evidente. O retrabalho, decorrente de funcionalidades implementadas em desacordo com o especificado, é um dos principais fatores de erosão de cronogramas e orçamentos. Portanto, a investigação de ferramentas que automatizam a verificação da conformidade, como proposto neste trabalho, representa uma abordagem proativa para mitigar riscos financeiros e temporais, reduzindo a incidência de correções tardias e onerosas.

\section{Qualidade de Software}

A qualidade de software é o resultado de um processo que integra planejamento, definição de métricas, verificação contínua e validação sistemática de requisitos. De acordo com \citeonline{sommerville2011}, a qualidade não se restringe ao produto final, mas depende fundamentalmente do processo adotado para desenvolvê-lo. De forma complementar, \citeonline{pressman2016} argumenta que a qualidade deve ser construída ao longo de todo o ciclo de vida e avaliada por meio de medições objetivas, em vez de depender de inspeções pontuais.


O estabelecimento de métricas é um processo colaborativo. O Guia PMBOK \textit{\cite{pmi2008}} aponta que a definição dos parâmetros de avaliação é uma responsabilidade compartilhada entre gestores, \textit{stakeholders} e desenvolvedores. Em projetos de maior porte, equipes de QA atuam de forma independente para assegurar a imparcialidade na validação. Em contextos menores, esse papel pode ser desempenhado pelo time de desenvolvimento,  embora \citeonline{sommerville2011} alerte que a falta de independência pode comprometer a objetividade da avaliação.

A validação da implementação de um requisito depende, primeiramente, da definição de critérios de aceitação claros, que descrevem em que condições a funcionalidade pode ser considerada concluída. Esses critérios são transformados em casos de teste, os quais simulam tanto cenários de uso esperados quanto situações de exceção.\citeonline{pressman2016} destaca que a determinação da quantidade de cenários de teste deve estar diretamente relacionada à complexidade do requisito: funcionalidades críticas, com múltiplas entradas ou impacto na segurança, exigem uma cobertura mais extensa do que funcionalidades simples e de baixo risco. \citeonline{sommerville2011} acrescenta que, em muitos casos, os testes devem abranger não apenas requisitos funcionais verificando se a função solicitada foi implementada corretamente, mas também requisitos não funcionais, como desempenho, confiabilidade e usabilidade, cuja validação demanda métricas específicas.

A medição da qualidade resulta, portanto, da combinação de diferentes perspectivas. Por um lado, existem métricas quantitativas, como a cobertura de requisitos (proporção de requisitos associados a testes), a densidade de defeitos (número de falhas encontradas por unidade de artefato) e o esforço médio para corrigir erros, amplamente discutidas por \citeonline{pressman2016}. Por outro lado, existem aspectos qualitativos, como a clareza da documentação, a consistência de interfaces e a facilidade de manutenção, que, ainda segundo Sommerville, exigem inspeções estruturadas e o julgamento de especialistas.

O Guia PMBOK \textit{\cite{pmi2008}} enfatiza que tais processos de avaliação devem ser acompanhados de forma contínua, envolvendo tanto a prevenção quanto a correção de falhas. A prevenção é garantida pelo planejamento da qualidade, quando se definem padrões, métricas e critérios de aceitação, já a correção ocorre nos processos de controle, quando os resultados de testes e inspeções são comparados com os padrões estabelecidos e são adotadas ações corretivas em caso de desvios. Assim, a definição de métricas, a escolha dos testes e a determinação da quantidade de cenários não são decisões arbitrárias, mas decorrem de um processo de gestão da qualidade que envolve múltiplos agentes, desde o cliente até a equipe técnica, e que se apoia em práticas consolidadas de engenharia de software.

Em síntese, a validação da implementação de um requisito não se limita a confirmar a presença de código correspondente, mas exige que sejam definidos critérios de aceitação verificáveis, transformados em cenários de teste, avaliados por meio de métricas objetivas e complementados por revisões independentes. A qualidade, portanto, emerge de um processo coletivo e interativo, que combina prevenção, medição e melhoria contínua, garantindo que o software entregue esteja não apenas em conformidade com as especificações, mas também adequado ao uso e sustentável em longo prazo.

\section{Fatores Críticos em Projetos de Software}

O êxito de projetos de software transcende a aplicação de técnicas e ferramentas, dependendo de um espectro de fatores organizacionais, humanos e gerenciais. \citeonline{sommerville2011} argumenta que falhas de comunicação estão entre as principais causas de insucesso, pois comprometem o alinhamento de expectativas e geram ambiguidades nos requisitos. O autor também destaca a ausência de um gerenciamento de riscos estruturado como um fator que reduz a resiliência do projeto.

O êxito de projetos de software não depende exclusivamente da aplicação de técnicas e ferramentas de desenvolvimento, mas também de uma ampla gama de fatores organizacionais, humanos e gerenciais. \citeonline{sommerville2011} argumenta que falhas de comunicação entre os membros da equipe e os \textit{stakeholders} figuram entre as principais causas de insucesso em projetos de software, uma vez que comprometem o alinhamento de expectativas, dificultam a tomada de decisões e favorecem o surgimento de ambiguidade nos requisitos. Além disso, o autor destaca que a ausência de práticas estruturadas de gerenciamento de riscos reduz a capacidade do projeto de responder adequadamente a imprevistos, afetando diretamente sua resiliência e previsibilidade.

Complementarmente, \citeonline{pressman2016} enfatiza a importância da motivação e da capacitação da equipe como elementos essenciais. Times engajados e tecnicamente preparados respondem com maior eficiência a pressões de prazo, escopo e desafios técnicos. O autor ressalta, ainda, que arquiteturas modulares e documentação consistente favorecem a sustentabilidade do software ao longo de seu ciclo de vida.

Outro aspecto relevante é a gestão do conhecimento. Ambientes que promovem a troca de experiências e a retroalimentação de processos apresentam melhores índices de produtividade e qualidade, especialmente em cenários de alta incerteza. Dessa forma, os fatores críticos de sucesso resultam da integração equilibrada entre pessoas, processos e tecnologia, exigindo uma abordagem holística do desenvolvimento.

Dessa forma, os fatores críticos de sucesso em projetos de software podem ser compreendidos como resultado da integração equilibrada entre pessoas, processos e tecnologia. Tal integração pressupõe uma abordagem holística do desenvolvimento de software, na qual aspectos técnicos não são tratados isoladamente, mas articulados com práticas gerenciais, estratégias de comunicação, definição clara de papéis e investimento contínuo em capacitação.

Em síntese, o sucesso de um projeto não é determinado apenas pela adoção de boas práticas técnicas, mas pela capacidade da organização de estruturar um ecossistema de desenvolvimento que favoreça a adaptação, o comprometimento coletivo e a gestão proativa dos fatores críticos que impactam diretamente o desempenho e a entrega de valor do software. Contudo, mesmo em ambientes bem estruturados, subsiste um desafio persistente: garantir que os requisitos formalmente definidos sejam de fato corretamente implementados, preservando sua intenção original ao longo do ciclo de vida do software. 

Ferramentas tradicionais, como \textit{linters} e analisadores estáticos, embora eficazes na detecção de erros sintáticos e estruturais no código-fonte, não possuem mecanismos capazes de interpretar a semântica contida em documentos de requisitos escritos em linguagem natural. Essa limitação compromete sua eficácia na verificação da conformidade entre especificações e implementações, especialmente em contextos complexos e sujeitos a variações terminológicas. É precisamente nessa lacuna que os LLMs se inserem como uma inovação promissora. Dotados de mecanismos avançados de representação contextual e capacidade de correlacionar textos descritivos com estruturas de código, esses modelos inauguram uma nova classe de ferramentas capazes de realizar análises semânticas mais profundas e alinhadas ao domínio da Engenharia de Requisitos. Assim, ao incorporar LLMs ao processo de verificação de conformidade, este trabalho busca explorar seu potencial para aprimorar a rastreabilidade, mitigar ambiguidades e reduzir o retrabalho decorrente de interpretações incorretas ou incompletas dos requisitos.
